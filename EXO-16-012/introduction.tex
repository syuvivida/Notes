\section{Introduction \label{sec:intro}}



Although more than 80\% of the matter in the Universe is dark matter (DM/$\chi$), its 
underlying nature remains unknown and cannot be explained within the standard 
model (SM). If non-gravitational interactions exist between dark matter and 
standard model particles, dark matter particles could be produced at the LHC. 
One way to observe them is when the dark matter particles are produced in 
association with a visible SM particle X (=g, q, $\gamma$, Z, W, or 
Higgs). 
%Such reactions, which are observed at colliders as particles or jets 
%recoiling against an invisible state, are called “mono-X” or \MET+X. 
%
%
The SM particle X could be emitted directly from a quark as initial state 
radiation (ISR) or as part of new effective vertex coupling of DM to SM. Unlike
W, Z, jet, or $\gamma$, Higgs ISR is highly suppressed, which implies that 
mono-Higgs signal could probe directly the structure of the effective DM-SM 
coupling. 
%This document describe the analysis of the search for dark matter candidate in 
%mono Higgs events using the 2015 LHC data collected by the CMS detector.

In this document, we present a search for dark matter in mono Higgs events 
via the decay mode $\Ph\rightarrow b\bar{b}$. This analysis is based on 
integrated luminosities of 2.2 \fbinv of pp collisions at $\sqrt{s}=13\TeV$, 
collected in 2015 with the CMS detector. 
%
%Our analysis is sub-divided into
%two regimes: (i) non-boosted regime where the Higgs boson is reconstructed by 
%two separate b-jets with a jet radius of $\Delta R=0.4$, and (ii) boosted 
%regime where the Higgs boson is reconstructed by one single jet with a jet 
%radius of $\Delta R = 0.8$ and jet substructure variables are employed to 
%improve the separation of signal and background.
%
At the LHC, searches for dark matter in mono Higgs events had been performed 
by the ATLAS Collaboration using 20~\fbinv of $\sqrt{s}=8\TeV$ data, 
via the decay mode $\Ph\rightarrow \gamma\gamma$~\cite{ATLASHAA} and 
$\Ph\rightarrow b\bar{b}$~\cite{ATLASHBB}. Limits were set as a function 
of DM mass and parameters in various DM models. In Run 1, the CMS 
Collaboration had not yet performed any mono-Higgs searches.


%This note is organized as follows: Section~\ref{sec:model} gives more details of the
%2HDM, Section~\ref{sec:dec} describes the CMS detector while Sections~\ref{sec:samples}--\ref{sec:sel} 
%list the data and simulated samples used in this analysis, the triggers, and the
%reconstruction of objects and event selection, respectively. Sections~\ref{sec:bkgcontrol} and \ref{sec:sig} 
%detail the estimation of background and the extraction of signal. 
%The systematic uncertainties are discussed in Section~\ref{sec:sys} while the results 
%and conclusion are presented in Sections~\ref{sec:results} and \ref{sec:con}.


%The interpretation of the dark matter signal is usually performed with
%two approaches: (i) effective field theories (EFTs), and (ii) simplified models. 
%
%The EFTs assume that DM couples directly to SM particles through a
%contact interaction involving a quark-antiquark pair, or two gluons,
%and two DM particles and the X particle. The contact interaction 
%is described by non-renormalizable n-dimensional operators. 
%Various EFTs operators for the mono-H search are introduced 
%and discussed in detail in Ref.~\cite{Irvine}.
%The EFTs are valid when the mediators of the interaction between SM and DM 
%particles are very heavy; if this is not the case, models that explicitly include 
%these mediators are needed (simplified models). 
%
%
%The ATLAS-CMS DM forum paper~\cite{ATLASCMS-DM} recommends three 
%benchmark simplified models for the mono-H searches. 
%\begin{itemize}
%\item A model where a vector mediator ($\cPZpr_{B}$ ) is exchanged in
%  the s-channel, radiates a Higgs boson, and decays into two DM
%  particles. In this model, the couplings of the $\cPZpr_{B}$ to
%  leptons are omitted.
%%
%\item A model where a scalar mediator $S$ is emitted from the Higgs
%  boson and decays to a pair of DM particles.
%%
%\item  A model where a vector \cPZpr\ is produced resonantly and
%  decays into a Higgs boson plus an intermediate heavy pseudoscalar
%  particle \Az, in turn decaying into two DM particles, see Fig.~\ref{fig:feynman}.
%\end{itemize}



%We interpret our data with a benchmark model: two Higgs doublet model 
%(2HDM)~\cite{2HDM}. 
%In this model, a vector \cPZpr\ is produced resonantly and decays into
%a Higgs boson plus an intermediate heavy pseudoscalar particle \Az, in
%turn decaying into two DM particles, see Fig.~\ref{fig:feynman}.

\begin{figure}[htbp]
   \centering
   \includegraphics[width=0.5\textwidth]{Figures/feynman_2HDM.pdf}
   \caption{Feynman diagram of a two higgs doublet model with a new
     invisibly decaying pseudoscalar \Az from the decay of an on-shell
     resonance \cPZpr\  giving rise to a Higgs+\MET signature. }
   \label{fig:feynman}
\end{figure}

We interpret our data with a benchmark simplified model: two Higgs doublet model 
(2HDM)~\cite{2HDM}. 
%For the first results, we focus on the two Higgs
%doublet model (2HDM)~\cite{2HDM}, 
where a vector \cPZpr\ is produced resonantly and 
decays into a Higgs boson plus an intermediate heavy pseudoscalar particle \Az, in turn decaying into two DM particles, see Fig.~\ref{fig:feynman}. 
%
%since the 2HDM is expected to produce, on average, a higher \MET value compared to the other two models due to
%the on-shell \cPZpr\ production, see Fig.~\ref{fig:METmodel}. 
%
A Type-2 two higgs doublet is assumed, where 
$\Phi_u$ couples to up-type quarks and $\Phi_d$ couples to down-type
quarks and leptons. The gauge symmetry of the SM is extended by a
$U(1)_{\cPZpr}$, with a new massive \cPZpr\ gauge boson. In this 2HDM
model, only $\Phi_u$  and right-handed up-type quarks $u_R$ are charged 
under the $U(1)_{\cPZpr}$ while $\Phi_d$ and  all the other SM
fermions are neutral. 

After electroweak symmetry breaking, the Higgs doublets attain vacuum 
expectation values $v_u$ and $v_d$, and in unitary gauge the doublets are
parametrized as

\[
\Phi_d = \frac{1}{\sqrt{2}}
\begin{pmatrix}
 -\sin\beta~\PH^+\\
v_d - \sin\alpha~\Ph + cos\alpha~\PH - i \sin\beta~\Az
\end{pmatrix},
\]


\[
\Phi_u = \frac{1}{\sqrt{2}}
\begin{pmatrix}
 -\cos\beta~\PH^+\\
v_u + \cos\alpha~\Ph + sin\alpha~\PH + i \cos\beta~\Az
\end{pmatrix},
\]

where \Ph, \PH\ are neutral CP-even scalars, \Hpm\ is a charged scalar,
and \Az\ is a neutral CP-odd scalar. In this framework,
$\tan\beta\equiv v_u/v_d$, and $\alpha$ is the mixing angle that diagonalizes
the $\Ph-\PH$ mass squared matrix. This model also contains an
additional scalar singlet $\phi$ that leads to spontaneous symmetry
breaking. The $\alpha$ is assigned to be $\alpha = \beta - \pi/2$, 
in the limit where the \Ph\ has SM-like couplings to fermions and gauge
bosons, and $\tan \beta \geq 0.3$ as implied from the perturbativity
of the top Yukawa coupling. 

The model is described by five parameters, namely, (i) the pseudoscalar mass $M_{\Az}$, (ii)the DM mass $M_{\chi}$, (iii) the \cPZpr\ mass $M_{\cPZpr}$,
(iv) $\tan \beta$, and (v) the  \cPZpr\ coupling strength $g_z$.
However, only the masses $M_{\Az}$ and $M_{\cPZpr}$ affect the 
kinematic distributions and all the other parameters affect the cross sections and decay widths only. 

%% following text is about final state

\par The  transverse momentum of the Higgs boson is directly proportional to the mass of the $Z`$ ($M_{\cPZpr}$). 
The spatial distance ($\Delta R = \sqrt{ \Delta\eta^2 + \Delta \phi^2  }$) 
between the decay products of the Higgs boson ($\b\bar{b}$) follows the relation $\Delta R = 2 \times M_{H}/p_{H}$. 
The present search analysis consider $M_{\cPZpr}$ ranging from 
600 GeV -- 2500 GeV which implies a very wide range of transverse momentum of the Higgs boson and $\Delta R(b\bar{b})$. 
Therefore the analysis is divided into two regimes : (i) non-boosted regime where the Higgs boson is reconstructed by two separate b-jets with a jet radius of $\Delta R =$ 0.4, (see section \ref{eventreco})
and (ii) boosted regime where the Higgs boson is reconstructed by one single jet with a jet radius $\Delta R =$ 0.8. 
%The efficiency to reconstruct the Higgs boson using to resolved AK04 jets drops when $M_{\cPZpr}$ $>$ 1000 GeV as the two b-jets start to merge into each other. 
%Therefore these merged jets are reconstructed using larger jet-size parameter (R=0.8), AK08 jets. 

%The decay width of $\cPZpr\rightarrow \Ph
%\Az$ is:
%\begin{equation}
%\Gamma_{\cPZpr\rightarrow\Ph\Az}= \left(g_z
%  \cos\alpha\cos\beta\right)^2 \frac{\left | p\right|}{24\pi}\frac{\left | p\right| ^2}{M_{\cPZpr}^2},
%\end{equation}
%where $|p|$ is the center of mass momentum for the decay products
%
%\[
% |p| = \frac{1}{2M_{\cPZpr}} \sqrt{
%   \left(M_{\cPZpr}^2-\left(m_{\Ph}+m_{\Az}\right)^2\right) 
%\left(M_{\cPZpr}^2-\left(m_{\Ph}-m_{\Az}\right)^2\right)}.
%\]
%
%
%The coupling $g_z$ is constrained by electroweak global fits and
%searches for dijet resonances from $\cPZpr\rightarrow q\bar{q}$:
%\begin{equation}
%g_z < 0.03 \cdot \frac{g}{\cos\theta_w \sin^2\beta}{\frac{\sqrt{M_{\cPZpr}^2 - M_Z^2}}{M_Z}.\label{eq:gz}
%\end{equation}
%The cross sections listed in Section~\ref{sec:sigSamp} were computed assuming the upper
%limit of $g_Z$ in Eqn~\ref{eq:gz}.
%
%
%\begin{figure}[htbp]
%   \centering
%   \includegraphics[width=0.4\textwidth]{Figures/signal-model/METhighM.pdf}
%   \includegraphics[width=0.4\textwidth]{Figures/signal-model/METlowM.pdf}
%   \caption{ Comparison of the \MET distributions at generator level
%     in different simplified models leading to a Higgs+\MET 
%       signature~\cite{ATLASCMS-DM}, for high mediator mass at 1 TeV
%       (left) and low mediator mass at 100 GeV (right). In both cases,
%     the DM particles are produced on-shell from the mediators.}
%   \label{fig:METmodel}
%\end{figure}
%



